% !TEX TS-program = xelatex
% !TEX encoding = UTF-8 Unicode
% !Mode:: "TeX:UTF-8"

\documentclass{resume}
\usepackage{zh_CN-Adobefonts_external} % Simplified Chinese Support using external fonts (./fonts/zh_CN-Adobe/)
%\usepackage{zh_CN-Adobefonts_internal} % Simplified Chinese Support using system fonts
\usepackage{linespacing_fix} % disable extra space before next section
\usepackage{cite}

\begin{document}
\pagenumbering{gobble} % suppress displaying page number

\name{张文杰}

\basicInfo{
  \phone{155-2754-3952} \textperiodcentered\ 
  \email{memory7734@gmail.com} \textperiodcentered\ 
  \github[memory7734]{https://github.com/memory7734}}
 
\section{\faGraduationCap\ 教育背景}
\datedsubsection{\textbf{华中科技大学}\  计算机技术\ 硕士研究生在读}{2018.9 - 至今}
\ 推荐免试入学,校级一等奖学金,智能与分布计算实验室优秀工程奖,预计2020年7月毕业
\datedsubsection{\textbf{武汉理工大学}\  软件工程\ 工学学士}{2014.9 - 2018.6}
\ \textbf{排名19/225(前10\%)},CET-4:497,CET-6:486,国家级大学生创新创业训练计划优秀项目,国奖励志奖学金(2次),校级三等奖学金,校级三好学生,校级优秀班干部,院级三好学生,武汉理工大学优秀毕业生

\section{\faUsers\ 实习经历}
\datedsubsection{\textbf{阿里巴巴-新零售技术事业群}}{2019.6 - 至今}
\begin{onehalfspacing}
\begin{itemize}
  \item todo
\end{itemize}
\end{onehalfspacing}

\datedsubsection{\textbf{南京吾道知信信息技术有限公司}}{2019.1 - 2019.6}
\begin{onehalfspacing}
\begin{itemize}
  \item 针对中国证监会、上海证券交易所、深圳证券交易所和中国债券信息网分别开发公司债部分的通用爬虫,对各地市的环保局、工商局官网进行指定内容的爬取
  \item 负责“公司债”类型的公告筛选、文本解析工作。对于10万多份PDF公告进行去重匹配,将其中的结构化和非结构化数据,通过规则将涉及26种类型的公告信息处理为结构化数据约600个字段,生成结果交内容团队校对
  \item 将解析处理流程与公司开发的校对系统整合,定时更新提取结果,内容团队可以直接通过浏览器完成校对,根据校对结果将公告提取出的信息自动生成时间轴,并与公司其它产品关联,将一支公司债的完整生命周期作为产品展示出来
\end{itemize}
\end{onehalfspacing}

\section{\faInbox\ 项目经历}

\datedsubsection{\textbf{基于分布式的可拔插计算调度框架}}{2018.4 - 2018.6}
\begin{onehalfspacing}
\begin{itemize}
  \item 采用主从结构,中心节点负责系统的调度与任务的分发工作,任务节点对任务进行计算并把结果反馈给中心节点。采用ZooKeeper技术进行服务注册与发现,使用Netty框架开发网络底层通信协议,对于二进制数据进行了序列化与反序列化的处理,同时使用远程过程调用保证对从节点的任务调用
  \item 结合优先级、定时任务、上传时间等多方面制定动态任务分配算法,中心节点根据动态分配算法将任务分发到任务节点。使用心跳检测机制中心节点可以实时的获取服务节点的运行状态,确保当任务节点出现单点故障时能够及时的进行任务重传保证系统具有一定的容错性
\end{itemize}
\end{onehalfspacing}

\section{\faPaperPlaneO\ 比赛经历}
\datedsubsection{\textbf{第五届中间件性能挑战赛}}{2019.6 - 2019.7}
\begin{onehalfspacing}
\begin{itemize}
  \item 自适应负载均衡的设计实现:基于动态调整的最小活跃线程数调整请求分发,有效测试时段60s内有效请求数超过114万,最大TPS超过2万
\end{itemize}
\end{onehalfspacing}

\section{\faStar\ 成果奖项}
\datedline{Moving\textit{健康管理系统},计算机软件著作权}{2017.6}
\datedline{\textit{第八届工信部“蓝桥杯”}Java\textit{软件开发大赛全国三等奖}}{2017.5}
% \section{\faCube\ 发表成果}

\end{document}
